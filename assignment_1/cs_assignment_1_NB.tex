\documentclass{article}

% Language setting
% Replace `english' with e.g. `spanish' to change the document language
\usepackage[english]{babel}

% Set page size and margins
% Replace `letterpaper' with `a4paper' for UK/EU standard size
\usepackage[a4paper,top=2cm,bottom=2cm,left=3cm,right=3cm,marginparwidth=1.75cm]{geometry}

% Useful packages
\usepackage{amsmath,amssymb,amsthm}
\usepackage{graphicx}
\usepackage[colorlinks=true, allcolors=blue]{hyperref}

\title{Assignment n°1 \\ Foundations of Machine Learning}
\author{Nicolas Bourriez}

\begin{document}
\maketitle

\section{Probability and statistics}

\subsection{Bayes theorem}

\newline Let's define some random variables to describe our problem:
\begin{itemize}
\item $I = Infected$
\item $S = Safe$
\item $PT = Positive Test$
\item $NT = Negative Test$
\end{itemize}
\newline and the associated probabilities that we know:
\begin{itemize}
\item $P(I) = 1/1000$
\item $P_S(PT) = 0.01$
\item $P_S(NT) = 0.99$
\end{itemize}
\newline We want to know the probability of a randomly tested person to be \textit{actually} infected if its test is \textit{positive} $P_{PT}(I)$.
\newline We recall that the Bayes theorem gives us the following equality:

%Bayes Theorem equation
\begin{equation} \label{eq1}
P_B(A)=\frac{P(A)P_A(B)}{P(B)} 
\end{equation}

%Demo
\newline so we can write
\[
P_{PT}(I)=\frac{P(I)P_I(PT)}{P(PT)} 
\]
However we notice that
\[
\begin{split}
P(PT) & = P_I(PT)P(I) + P_S(PT)P(S) \\
 & = 0.95\times0.001 + 0.01\times0.999 \\
 & = \frac{1}{1000} 
\end{split}
\]
So we write
\[
\begin{split}
P_{PT}{(I)} & = \frac{P(I)P_I(PT)}{0.011} \\
 & = \frac{0.001\times0.95}{0.011} \\
 & \simeq \frac{8.6}{1000}
\end{split}
\]
\textbf{CONCLUSION}: we find that the probability of a randomly tested person to be \textit{actually} infected if its test is \textit{positive} is equivalent to:
\[
\boxed{
 & \simeq \frac{8.6}{1000}
}
\]

\subsection{Maximum Likelihood Estimator}
\subsubsection{Log-likelihood of the data}
Given the N independent and identically distributed $(i.i.d)$ samples $\mathbf{x} = {x_1,x_2,...,x_N}$, we can write the \textit{likelihood} of the data as

\begin{equation} \label{eq2}
p(\mathbf{x}|\theta)=\mathcal{L}_{\theta}(\mathbf{x})=\prod\limits_{i=1}^{n}p(x_i|\theta) \quad\quad\quad (i.i.d)
\end{equation}

Using the natural logarithm $\ln$, we thus write
\[
\begin{split}
l(\theta) = \ln(\mathcal{L}_{\theta}) & = \ln(\prod\limits_{i=1}^{n}p(x_i|\theta)) \\
& = \sum\limits_{i=1}^{n}\ln(p(x_i|\theta)) \\
& = \sum\limits_{i=1}^{n}\ln(\frac{1}{\sqrt{2\pi\sigma^2}}e^{-\frac{1}{2}\frac{(x_i-\theta)^2}{\sigma^2}}) \quad\quad\quad (w.r.t \quad \mu)
\end{split}
\]
We thus have the following equation using properties of $\ln$
\[
\begin{split}
l(\theta) & = \sum\limits_{i=1}^{n}{(-\ln\sqrt{2\pi\sigma^2} + \ln e^{-\frac{1}{2}\frac{(x_i-\theta)^2}{\sigma^2}})} \\
& = - \frac{1}{2\sigma^2}\sum\limits_{i=1}^{n}(x_i-\theta)^2-\sum\limits_{i=1}^{n}\frac{\ln(2\pi\sigma^2)}{2}
\end{split}
\]

\textbf{CONCLUSION}:
\begin{equation} \label{eq3}
\boxed{
l(\theta) = - \frac{1}{2\sigma^2}\sum\limits_{i=1}^{n}(x_i-\theta)^2-\frac{n}{2}\ln(2\pi\sigma^2)
}
\end{equation}

\subsubsection{Maximum Likelihood Estimator of $\mu$}

We define $\hat{\theta}({\mathbf{x}})$ as the Maximum Likelihood Estimator of $\mu$. It is thus the maximum argument of the likelihood $\mathcal{L}_{\theta}(\mathbf{x})$, which means that we want to take the likelihood for which the parameters $\theta$ provide the "highest" probabilities.
\[
\hat{\theta}({\mathbf{x}}) = \underset{\theta}{argmax}\quad{\mathcal{L}_\theta(\mathbf{x})}
\]
Since we want to maximize our probability, we want to reach an extremum, and thus we can use the derivative of our \textit{log-likelihood} by setting it to 0:
\[
\hat{\theta}= \underset{\theta}{argmax}\quad l(\theta) \longrightarrow \frac{\partial l}{\partial \theta} = 0 \quad\quad\quad (w.r.t \quad \mu)
\]
We thus first compute the partial derivative of l  w.r.t $\theta$, which can be written as being proportional to an easier expression from .Equation (3)
\[
\begin{split}
\frac{\partial l(\theta)}{\partial \theta} & = \frac{\partial l}{\partial \theta}\left[- \frac{1}{2\sigma^2}\sum\limits_{i=1}^{n}(x_i-\theta)^2-\frac{n}{2}\ln(2\pi\sigma^2))\right] \\
\frac{\partial l(\theta)}{\partial \theta} & \propto \sum\limits_i{(\theta-x_i)} \\
& \propto n\theta - \sum\limits_i{x_i}
\end{split}
\]
Thus we can write:
\[
\begin{split}
\frac{\partial l}{\partial \theta} = 0 & \iff n\theta - \sum\limits_i{x_i} = 0 \\
& \iff \theta = \frac{1}{n}\sum\limits_{i}x_i
\end{split}
\]

\textbf{CONCLUSION}: We find that the Maximum Likelihood Estimator of $\mu$ is equal to the mean $\overline{X}_n$
\begin{equation} \label{eq3}
\boxed{
\hat{\theta} = \underset{\theta}{argmax}l(\theta) = - \frac{1}{n}\sum\limits_{i}x_i
}
\end{equation}

\subsubsection{Maximum Likelihood Estimator of $\sigma^2$}

We define $\hat{\theta}({\mathbf{x}})$ as the Maximum Likelihood Estimator of $\sigma^2$.
\newline Similarly to 1.2.2, we can write the log-likelihood of $\theta$ with respect to $\sigma^2$:
\[
\begin{split}
l(\theta) & = \sum\limits_{i=1}^{n}\ln(\frac{1}{\sqrt{2\pi\theta}}e^{-\frac{1}{2}\frac{(x_i-\mu)^2}{\theta}}) \quad\quad\quad (w.r.t \quad \sigma^2) \\
& = - \frac{1}{2\theta}\sum\limits_{i=1}^{n}(x_i-\mu)^2-\frac{n}{2}\ln(2\pi\theta)
\end{split}
\]
Same as for the MLE of $\mu$, here we want to find the right $\sigma^2$ that will maximize the likelihood $\mathcal{L}_\theta(\mathbf{x})$.
\newline We thus have:
\[
\hat{\theta}= \underset{\theta}{argmax}\quad l(\theta) \longrightarrow \frac{\partial l}{\partial \theta} = 0 \quad\quad\quad (w.r.t \quad \sigma^2)
\]
Computing the partial derivative of l, we get:

\[
\begin{split}
\frac{\partial l}{\partial \theta} & = \frac{\left[ -\sum\limits_{i=1}^{n}(x_i -\mu)^2\right]'2\theta - \left[ -\sum\limits_{i=1}^{n}(x_i -\mu)^2\right](2\theta)'}{(2\theta)^2} - \left[(\frac{n}{2})'\ln(2\pi\theta) + \frac{n}{2}\ln(2\pi\theta)'\right] \\
& = \frac{\sum\limits_{i=1}^{n}(x_i-\mu)^2}{2\theta^2}-\frac{n}{2\theta} \\
& = \frac{\sum\limits_{i=1}^{n}(x_i-\mu)^2 - n\theta}{2\theta^2} \quad\quad\quad (\theta \neq  0)
\end{split}
\]
We pose 
\[
\hat{\theta}= \underset{\theta}{argmax}\quad l(\theta) \longrightarrow \frac{\partial l}{\partial \theta} = 0 \quad\quad\quad (w.r.t \quad \sigma^2)
\]
And thus we can then compute the MLE of $\sigma^2$ which gives:

\[
\begin{split}
\hat{\theta} = \underset{\theta}{argmax}l(\theta) & \longrightarrow \frac{\sum\limits_{i=1}^{n}(x_i-\mu)^2-n\theta}{2\theta} = 0  \\
& \longrightarrow \sum\limits_{i=1}^{n}(x_i-\mu)^2-n\theta = 0 \\
& \longrightarrow \theta = \frac{1}{n}\sum\limits_{i = 1}^{n}(x_i-\mu)^2
\end{split}
\]

\textbf{CONCLUSION}: We find that the Maximum Likelihood Estimator of $\sigma^2$ is equal to :
\begin{equation} \label{eq3}
\boxed{
\hat{\theta} = \underset{\theta}{argmax}l(\theta) = \frac{1}{n}\sum\limits_{i}(x_i-\mu)^2
}
\end{equation}

\subsubsection{Bonus question}

We recall that an unbiased estimator $\hat{\theta}$ is equal to $\mathbb{E}(\hat{\theta})=\theta$.
\newline We want to show that the Maximum Likelihood Estimator of $\sigma^2$ is \textbf{biased}.
\newline From .Equation (5), we already have the following expression:

\[
\hat{\theta} = \frac{1}{n}\sum\limits_{i=1}^{n}(x_i - \mu)^2
\]

Moreover since we recall from .Equation (4) that $\hat{\theta}$ with respect to $\mu$ is the \textit{\textbf{mean}} $\overline{X}_n$, we can write:

\[
\begin{split}
\hat{\theta} & = \frac{1}{n}\sum\limits_{i=1}^{n}(X_i - 2X_i\overline{X}_n+\overline{X}_n^2) \\
& = \frac{1}{n}(\sum\limits_{i=1}^{n}X_i^2 - 2\overline{X}_n\sum\limits_{i=1}^{n}X_i+\sum\limits_{i=1}^{n}\overline{X}_n^2) \\
& = \frac{1}{n}(\sum\limits_{i=1}^{n}X_i^2 - 2n\overline{X}_n^2+n\overline{X}_n^2) \quad\quad\quad (n\overline{X_n} = \sum\limits_{i=1}^{n}X_i)
\end{split}
\]
Using the expression we just found, and using the linearity of the Expected Value, we get:
\[
\mathbb{E}(\hat{\theta})=\frac{1}{n}\sum\limits_{i=1}^{n}\mathbb{E}(X_i^2)-\mathbb{E}(\overline{X}_n^2)
\]

Let's look further into each element of $\mathbb{E}(\hat{\theta})$:
\begin{itemize}
\item \textbf{$\mathbb{E}(X_i^2)$} ? \\
\newline We know that 
\[
\begin{split}
\sigma^2 & = \mathbb{V}(X_i) = \mathbb{E}((X_i - \mathbb{E}(X_i)^2) \\
& = \mathbb{E}(X_i^2)- \mathbb{E}(X_i)^2
\end{split}
\]
Thus we end having:
\[
\mathbb{E}(X_i^2) = \sigma^2 + \mu^2
\]
And similarly
\[
\mathbb{E}(\overline{X}_n) = \mu
\]
\item \textbf{$\mathbb{E}(\overline{X}_n^2)$} ? \\
\newline We know that:
\[
\begin{split}
\mathbb{V}(\overline{X}_n) & = \frac{1}{n^2}\sum\limits_{i=1}^{n}\mathbb{V}(X_i) \\
& = \frac{1}{n^2}\sum\limits_{i=1}^{n}\sigma^2 \\
& = \frac{1}{n^2}\timesn\sigma^2 \\
& = \frac{\sigma^2}{n}
\end{split}
\]
We can now compute $\mathbb{E}(\overline{X}_n^2)$:

\[
\begin{split}
\mathbb{E}(\overline{X}_n^2) & = \mathbb{V}(\overline{X}_n) + (\mathbb{E}(\overline{X}_n))^2 \\
& = \frac{\sigma^2}{n} + \mu^2
\end{split}
\]
\end{itemize}
\newline Aggregating the two expressions we got, we can now formulate $\mathbb{E}(\hat{\theta})$:
\[
\begin{split}
\mathbb{E}(\hat{\theta}) & =\frac{1}{n}\sum\limits_{i=1}^{n}\mathbb{E}(X_i^2)-\mathbb{E}(\overline{X}_n^2) \\
& = \frac{1}{n}\sum\limits_{i=1}^{n}(\sigma^2+\mu^2)-\frac{\sigma^2}{n}+\mu^2 \\
& = \frac{1}{n}\times n(\sigma^2+\mu^2) - \frac{\sigma^2}{n} - \mu^2 \\
& = \sigma^2 + \mu^2 + \frac{\sigma^2}{n} - \mu^2 \\
& = \frac{n-1}{n}\sigma^2
\end{split}
\]

\textbf{CONCLUSION}:$\mathbb{E}(\hat{\theta})$ is a \textit{\textbf{biased}} estimator of $\sigma^2$ since $\mathbb{E}(\hat{\theta}) \neq \sigma^2$ , and its bias is equal to :
\[
\begin{split}
b & = \mathbb{E}(\hat{\theta}) -  \theta \\
& = \frac{n-1}{n}\sigma^2-\sigma^2
\end{split}
\]

\begin{equation} \label{eq3}
\boxed{
b = -\frac{\sigma^2}{n}
}
\end{equation}

\section{Linear Regression}
\subsection{Parameters of a linear regression}
\subsubsection{Derivative of w}
We have $\mathbf{w} = (w_1,w_2)^T = \begin{pmatrix}w_1\\w_2\end{pmatrix}$ so $\hat{y}_n=w_1x_{1,n}+w_2x_{2,n}$
\newline We know from the exercise details that our loss function that we want to derive is the following:
\[
J(\mathbf{w}) = \frac{1}{N}\sum\limits_{n=1}^{N}(y_n-\hat{y}_n)^2
\]
\newline We now compute the partial derivatives of J:
\begin{itemize}
\item $\frac{\partial J(\mathbf{w})}{\partial w_1}$
\[
\begin{split}
\frac{\partial J(\mathbf{w})}{\partial w_1} & = \frac{1}{N}\sum\limits_{n=1}^{N}\frac{\partial}{\partial w_1}(y_n-\hat{y}_n)^2 \\
\end{split}
\]
Focusing only on the term $\frac{\partial}{\partial w_1}(y_n-\hat{y}_n)^2$:
\[
\begin{split}
\frac{\partial}{\partial w_1}(y_n-\hat{y}_n)^2 & = (2y_n-\hat{y}_n)\frac{\partial}{\partial w_1}(y_n-\hat{y}_n) \quad\quad\quad (chain \quad rule) \\
& = (2y_n-\hat{y}_n)\times (-x_{1,n}) \quad\quad\quad (-x_{1,n} = \frac{\partial}{\partial w_1}(y_n-w_1x_{1,n}-w_2x_{2,n})
\end{split}
\]
We end up with the following expression:
\[
\begin{split}
\frac{\partial J(\mathbf{w})}{\partial w_1} & = \frac{2}{N}\sum\limits_{n=1}^{N}(y_n-w_1x_{1,n}-w_2x_{2,n})\times -x_{1,n} \\
& = \frac{2}{N}(w_1\sum\limits_{n=1}^{N}x_{1,n}^2+w_2\sum\limits_{n=1}^{N}x_{1,n}x_{2,n}-\sum\limits_{n=1}^{N}x_{1,n}y_{n})
\end{split}
\]
\item $\frac{\partial J(\mathbf{w})}{\partial w_2}$
Similarly, we get:
\[
\begin{split}
\frac{\partial J(\mathbf{w})}{\partial w_1} & = \frac{2}{N}\sum\limits_{n=1}^{N}(y_n-w_1x_{1,n}-w_2x_{2,n})\times -x_{2,n} \\
& = \frac{2}{N}(w_1\sum\limits_{n=1}^{N}x_{2,n}^2+w_2\sum\limits_{n=1}^{N}x_{1,n}x_{2,n}-\sum\limits_{n=1}^{N}x_{2,n}y_{n})
\end{split}
\]
\end{itemize}
\newline To simplify the notation, we can write the following terms as so:
\begin{itemize}
\item $t_1 = \sum_{n=1}^{N}x_{1,n}^2$ 
\item $t_2 = \sum_{n=1}^{N}x_{2,n}^2$
\item $s = \sum_{n=1}^{N}x_{1,n}x_{2,n}$
\item $v_1 = \sum_{n=1}^{N}x_{1,n}y_n$
\item $v_2 = \sum_{n=1}^{N}x_{2,n}y_n$
\end{itemize}
And get
\[
\begin{split}
\frac{\partial J(\mathbf{w})}{\partial \mathbf{w}} = 0 & \Longleftrightarrow \begin{cases}
\frac{\partial J(\mathbf{w})}{\partial w_1} = 0 \\
\frac{\partial J(\mathbf{w})}{\partial w_2} = 0 
\end{cases} \\
& \Longleftrightarrow
\begin{cases}
w_1 = \frac{t_2v_1-sv_2}{t_1t_2-s^2} \\
w_2 = \frac{t_1v_2-sv_2}{t_1t_2-s^2}
\end{cases}
\end{split}
\]

\textbf{CONCLUSION}: The derivative of $\mathbf{w}$ when set to 0 is equal to :

\begin{equation} \label{eq3}
\boxed{
\frac{\partial J(\mathbf{w})}{\partial \mathbf{w}} = 0 \Longleftrightarrow
\begin{cases}
w_1 = \frac{t_2v_1-sv_2}{t_1t_2-s^2} \\
w_2 = \frac{t_1v_2-sv_2}{t_1t_2-s^2}
\end{cases}
}
\end{equation}
\subsubsection{Compatibility with closed-form solution}
Since we are in dimension 2, we can see that $\mathbf{X} = (\sum\limits_{n=1}^{N}x_{1,n} \sum\limits_{n=1}^{N}x_{2,n})$ and thus we get:

\[
\mathbf{X^TX} = \begin{pmatrix}\sum\limits_{n=1}^{N}x_{1,n}^2&\sum\limits_{n=1}^{N}x_{1,n}x_{1,n}\\ 
\sum\limits_{n=1}^{N}x_{1,n}x_{1,n}&\sum\limits_{n=1}^{N}x_{2,n}^2
\end{pmatrix} 
\]
Using the formula to compute the inverse of a matrix, we can find:
\[
\mathbf{X^TX}^{-1} = \frac{1}{t_1t_2-s^2}
\begin{pmatrix}t_2&-s\\-s&t_1
\end{pmatrix} 
\]
\textbf{CONCLUSION}: Multiplying by $\mathbf{X^Ty}$, we end up with the following equation:

\begin{equation} \label{eq4}
\boxed{
\mathbf{w} = (\mathbf{X^TX})^{-1}\mathbf{X^Ty}=\frac{1}{t_1t_2-s^2}
\begin{pmatrix}t_2v_1-sv_2\\t_1v_2-cv_1
\end{pmatrix}
}
\end{equation}
which appear coherent

\subsection{Least Square Loss}

\subsubsection{Log-likelihood of the observations}
\begin{itemize}
\item \textbf{Likelihood}
\newline We know that $D = {(x_1,y_1),...,(x_n,y_n)}$, thus we can write the likelihood of $D$ given a set of parameters $\theta = (w, \sigma^2)$ as
\[
p(D|\theta) = \prod\limits_{n=1}^Np(x_n,y_n|w,\sigma^2) \quad\quad\quad (i.i.d)
\]
We recall that $p(a,b) = p(a|b)p(b)$, hence
\[
p(D|\theta) = \prod\limits_{n=1}^Np(x_n|y_n,w,\sigma^2)p(y_n,w,\sigma^2)
\]
But we also recall according to the compound probability theorem that
\[
p(A\cap B) = p(A|B)p(B) = p(B|A)p(A)
\]
We can thus write the likelihood as so:
\[
p(D|\theta) = \prod\limits_{n=1}^Np(y_n|x_n,w,\sigma^2)p(x_n,w,\sigma^2)
\]
\item\textbf{ Log-likelihood}
\newline We wish to write the log-likelihood of our observations D as a function of $w$.
\newline Using the natural log:
\[
\begin{split}
l(w) & = \ln\left[ \prod\limits_{n=1}^Np(y_n|x_n,w,\sigma^2)p(x_n,w,\sigma^2) \right] \\
& = \sum\limits_{i=1}^{N}\left[\ln(p(y_n|x_n,w,\sigma^2))+\ln(p(x_n|w,\sigma^2))\right]
\end{split}
\]
Analysing each term of the sum:
\begin{itemize}
\item $\sum\limits_{i=1}^{N}\ln(p(x_n|w,\sigma^2)) = \sum\limits_{i=1}^{N}\ln[\frac{1}{\sqrt{2\pi\sigma^2}}e^{(-\frac{(x_n-w)^2}{2\sigma^2})}] = -N\ln(\sqrt{2\pi\sigma^2})-\frac{1}{\sigma^2}\frac{1}{2}\sum\limits_{n=1}^{N}(x_n-w)^2$
\item $\sum\limits_{i=1}^{N}\ln(p(y_n|x_n,w,\sigma^2)) = 
\sum\limits_{i=1}^{N}\ln[\frac{1}{\sqrt{2\pi\sigma^2}}e^{(-\frac{(y_n -w^Tx_n)^2}{2\sigma^2})}] = -N\ln(\sqrt{2\pi\sigma^2})-\frac{1}{\sigma^2}\frac{1}{2}\sum\limits_{n=1}^{N}(y_n -w^Tx_n)^2$
\end{itemize}
\end{itemize}

\textbf{CONCLUSION}:

\begin{equation} \label{eq4}
\boxed{
l(w) = -2N\ln(\sqrt(2\pi\sigma^2)-\frac{1}{2\sigma^2}\sum\limits_{n=1}^{N}(x_n-w)^2-\frac{1}{2\sigma^2}\sum\limits_{n=1}^{N}(y_n-w^Tx_n)^2
}
\end{equation}

\subsubsection{Maximization of log-likelihood}
We notice first that due to \textbf{monotony} of $\ln$, we have:
\[
\underset{w}{argmax}\quad p(D|w,\sigma^2) \Longrightarrow \underset{w}{argmax} \ln(p(D|w,\sigma^2))
\]
Since we want to maximize our log-likelihood with respect to ${w}$, we can set:
\[
\hat{w} = \underset{w}{argmax}\quad l(w) \Longrightarrow \frac{\partial l(w)}{\partial w} = 0
\]
Computing the partial derivative of $l(w)$:
\[
\frac{\partial l(w)}{\partial w} = \frac{\partial}{\partial w}\left[ -2N\ln(\sqrt(2\pi\sigma^2)-\frac{1}{2\sigma^2}\sum\limits_{n=1}^{N}(x_n-w)^2-\frac{1}{2\sigma^2}\sum\limits_{n=1}^{N}(y_n-w^Tx_n)^2\right]
\]
\begin{itemize}
\item $-2N\ln(\sqrt(2\pi\sigma^2)$ is a constant, so its derivative is equal to 0
\item $\frac{\partial}{\partial w}\left[-\frac{1}{2\sigma^2}\sum\limits_{n=1}^{N}(x_n-w)^2\right] \propto Nw - \sum\limits_{n}x_n$ (from Exercise 1.2.2)
\end{itemize}
We arrive then to the following expression:
\[
\begin{split}
\frac{\partial l(w)}{\partial w} = 0 & \Longleftrightarrow Nw - \sum\limits_{n}x_n - \frac{\partial}{\partial w}\left[\frac{1}{2\sigma^2}\sum\limits_{n=1}^{N}(y_n-w^Tx_n)^2\right] \\
& \Longleftrightarrow N\left[w-\frac{1}{N}\sum\limits_{n}x_n-\frac{\partial}{\partial w}\left[\frac{1}{2\sigma^2}\frac{1}{N}\sum\limits_{n}(y_n-w^Tx_n)^2\right]\right]
\end{split}
\]
\textbf{CONCLUSION}: We recognize that to \textit{maximize} our log-likelihood, we have to \textit{minimize} the following term, which is the sum of the squared errors:

\begin{equation} \label{eq4}
\boxed{
\frac{1}{N}\sum\limits_{n}(y_n-w^Tx_n)^2
}
\end{equation}


\section{Logistic Regression}
\subsection{Link between odd ratio and sigmoid function}
As per its definition, a \textit{sigmoid function} is represented as follows
\[
\sigma :x \mapsto \frac{1}{1 + e^{-x}}
\]
We want to prove that using a linear model $\mathbf{w^Tx}$,
\[
\sigma(\mathbf{w^Tx})=\frac{1}{1 + e^{-\mathbf{w^Tx}}} \Longleftrightarrow \mathbf{w^Tx} \sim \log(\frac{p(y=1)}{p(y=0)})
\]
Starting from the right-hand side of the equivalence and applying the exp on the natural logarithm, we have:
\[
\begin{split}
& \Longleftrightarrow e^{\mathbf{w^Tx}} \sim \frac{p(y=1)}{p(y=0)}\\
& \Longleftrightarrow \frac{1}{e^{\mathbf{w^Tx}}} \sim \frac{p(y=0)}{p(y=1)} \\
& \Longleftrightarrow 1 + \frac{1}{e^{\mathbf{w^Tx}}}\sim 1 + \frac{p(y=0)}{p(y=1)} \\
& \Longleftrightarrow 1 + e^{(-\mathbf{w^Tx})} \sim \frac{p(y=1)+p(y=0)}{p(y=1)} \\
& \Longleftrightarrow \frac{1}{1 + e^{\mathbf{-w^Tx}}} \sim \frac{p(y=1)}{p(y=1) + p(y=0)}
\end{split}
\]

\textbf{CONCLUSION}: We observe that predicting the log of the odd ratio with a linear model is equivalent to predicting the proba of \textit{winning} by applying a sigmoid function to the output of a linear model, which translates to:

\begin{equation} \label{eq4}
\boxed{
\frac{1}{1 + e^{\mathbf{w^Tx}}} \sim \frac{p(y=1)}{p(y=1) + p(y=0)}
}
\end{equation}

\subsection{Link between maximization of log-likelihood and logistic regression model}
Given a set of observations $D = \{ (x_1,y_1),...,(x_n,y_n) \}$, we can write the likelihood of this set as
\[
p(D) = \prod\limits_{n=1}^{N}p(x_n,y_n) \quad\quad\quad (i.i.d)
\]
We recall that $p(a,b) = p(a|b)p(b)=p(b|a)p(a)$, hence
\[
p(D) = \prod\limits_{n=1}^Np(y_n|x_n)p(x_n)
\]
Observing the odd ratio, we also recognize that the probability mass function of the random variable X is the one of a Bernoulli variable:
\[
P(X=x) = 
\begin{cases}
p \quad if \quad x = 1 \\
1-p \quad if \quad x = 0 \\
0 \quad otherwise
\end{cases} \\
\]
which also translate into the following expression:
\[
\begin{split}
P(X=x) & = p^x(1-p)^{(1-x)}, \quad x \in \left{0,1\right} \\
& = e^{x\ln p}e^{(1-p)\ln(1-x)} \quad\quad\quad (a^n=e^{n\ln(a)})
\end{split}
\]
\newline From .Equation (11), we have 
\[
P(Y=1)=\frac{1}{1+e^{-\mathbf{w^Tx}}}
\]
Using the probability distribution we observed previously, we get
\[
P(Y=y) = 
\begin{cases}
\frac{1}{1+e^{-\mathbf{w^Tx}}} \quad if \quad y = 1 \\
1 - \frac{1}{1+e^{-\mathbf{w^Tx}}} \quad if \quad y = 0 \\
0 \quad otherwise
\end{cases} \\
\]
\textbf{Partial conclusion:} the PMF of $y$ \textit{\textbf{depends}} on $\mathbf{w}$.
\newline
\newline We now apply the natural logarithm to get the \textit{log-likelihood} of the set of observations $D$:
\[
\begin{split}
\ln(p(D)) & = \ln\prod\limits_{n=1}^{N}p(y_n|x_n)p(x_n) \\
\ln(p(D)) & = \sum\limits_{n=1}^{N}\ln(p(y_n|x_n))+ \sum\limits_{n=1}^{N}\ln(p(x_n)) 
\end{split}
\]
If we decompose each term of our expression, we then have
\begin{itemize}
    \item For $\sum\limits_{n=1}^{N}\ln(p(y_n|x_n))$
    \[
    \begin{split}
        \sum\limits_{n=1}^{N}\ln(p(y_n|x_n)) & = \sum\limits_{n=1}^{N}\ln(e^{y_n\ln x_n}e^{(1-x_n)\ln(1-y_n)}) \\
        & = \sum\limits_{n=1}^{N}\left[ y_n\ln x_n + (1-x_n)\ln(1-y_n)\right]
    \end{split}
    \]
    \item For $\sum\limits_{n=1}^{N}\ln(p(x_n))$, similarly we get
    \[
    \begin{split}
        \sum\limits_{n=1}^{N}\ln(p(x_n)) & = \sum\limits_{n=1}^{N}\ln(p(x_n|p)) \\
        & = \sum\limits_{n=1}^{N}\ln(e^{x_n\ln p}e^{(1-x_n)\ln(1-p)}) \\
        & = \sum\limits_{n=1}^{N}\left[x_n\ln p + (1-p)\ln(1-x_n)\right]
    \end{split}
    \]
    \newline We can recall that maximizing the \textit{log-likelihood} of $p(D)$ gives us the following expression:
    \[
    \underset{w}{argmax}\quad \ln(p((D)) \Longrightarrow \frac{\partial \ln(p((D))}{\partial w} = 0
    \]
    with the partial derivative of the \textit{log-likelihood} being 
    
    \begin{multline*}
    \frac{\partial \ln(p((D))}{\partial w} = \frac{\partial}{\partial w}\left[\sum\limits_{n=1}^{N}y_n\ln x_n (1-x_n)\ln(1-y_n)\right] \\ + \\ \frac{\partial}{\partial w}\left[\sum\limits_{n=1}^{N}x_n\ln p + (1-p)\ln(1-x_n)\right] \quad (\lambda f + \mu g)'=\lambda f' + \mu g')
    \end{multline*}
    
    \newline But we remember that only the PMF of $y$ depends on $\mathbf{w}$, so we deduct that the partial derivative $\frac{\partial}{\partial \mathbf{w}}$ of $\sum\limits_{n=1}^{N}\ln(p(x_n))$ is a \textbf{\textit{constant}}.


\textbf{CONCLUSION}: We then recognize that we have a logistic regression model and its corresponding \textit{training loss} by minimizing the following expression:

\begin{equation} \label{eq4}
\boxed{
J(\mathbf{w}) = - \sum\limits_{n=1}^{N}\left[y_n\ln x_n (1-x_n)\ln(1-y_n)\right] = - \sum\limits_{n=1}^{N}\ln(p(y_n|x_n))
}
\end{equation}
 
\end{itemize}

\section{Clustering}
\subsection{K-Means}
\begin{align}
If \quad a_{n,k} \in k, then \quad a_{n,k} & = 1 \\
else \quad a_{n,k} & = 0
\end{align}

Since we want to minimize the inertia, we want that:
\[
\frac{\partial J(c_k)}{\partial c_k} = 0
\]
Writing the partial derivative of $J(c_k)$:
\[
\begin{split}
\frac{\partial J(c_k)}{\partial c_k} & = \frac{1}{N}\frac{\partial }{\partial c_k}\left[\sum\limits_{n=1}^{N}\sum\limits_{k=1}^{N}a_{n,k}\parallel \mathbf{x}_n - \mathbf{c}_k \parallel_2 ^2\right]\\
& = \frac{1}{N}\frac{\partial }{\partial c_k}\left[\sum\limits_{n=1}^{N}\sum\limits_{k=1}^{N}a_{n,k}(\mathbf{x}_n - \mathbf{c}_k)^T(\mathbf{x}_n - \mathbf{c}_k)\right] \\
& = \frac{1}{N}\frac{\partial }{\partial c_k}\left[\sum\limits_{n=1}^{N}\sum\limits_{k=1}^{N}a_{n,k}(\mathbf{x}_n^T\mathbf{x}_n - \mathbf{x}_n^T\mathbf{c}_k - \mathbf{c}_k^T\mathbf{x}_n + \mathbf{c}_k^T\mathbf{c}_k)\right] \\
& = \frac{1}{N}\sum\limits_{n=1}^{N}a_{n,k}\frac{\partial }{\partial c_k}(\mathbf{x}_n^T\mathbf{x}_n - \mathbf{x}_n^T\mathbf{c}_k - \mathbf{c}_k^T\mathbf{x}_n + \mathbf{c}_k^T\mathbf{c}_k) \quad\quad\quad (Eq.(14): a_{n,k} \notin k \rightarrow a_{n,k} = 0) \\
\end{split}
\]
Because $(uv)'=(u'v + uv')$ and $\frac{\partial(x^Ta)}{\partial a} = \frac{\partial(a^Tx)}{\partial a} = a$, we thus get:
\[
\frac{\partial J(c_k)}{\partial c_k} = \frac{1}{N}\sum\limits_{n=1}^{N}a_{n,k}(-2\mathbf{x}_n +2\mathbf{c}_k)
\]
We can now compute:
\[
\begin{split}
\frac{\partial J(c_k)}{\partial c_k} = 0 & \Longleftrightarrow \frac{1}{N}\sum\limits_{n=1}^{N}a_{n,k}(-2\mathbf{x}_n +2\mathbf{c}_k) = 0\\
& \Longleftrightarrow \frac{2}{N}\sum\limits_{n=1}^{N}(-a_{n,k}\mathbf{x}_n +a_{n,k}\mathbf{c}_k) = 0 \\
& \Longleftrightarrow \frac{2}{N}\left[\sum\limits_{n=1}^{N}a_{n,k}\mathbf{c}_k-\sum\limits_{n=1}^{N}a_{n,k}\mathbf{x}_n\right]  = 0 \\
& \Longleftrightarrow \mathbf{c}_k = \frac{\sum\limits_{n=1}^{N}a_{n,k}\mathbf{x}_n}{\sum\limits_{n=1}^{N}a_{n,k}}
\end{split}
\]

\textbf{CONCLUSION}: When minimizing $J$, cluster centers ${c_1,...,c_K}$ are the means of the points assigned to the respective clusters as described by:

\begin{equation} \label{eq4}
\boxed{
\frac{\partial J(c_k)}{\partial c_k} = 0 \Longleftrightarrow \mathbf{c}_k = \frac{\sum\limits_{n=1}^{N}a_{n,k}\mathbf{x}_n}{\sum\limits_{n=1}^{N}a_{n,k}}
}
\end{equation}

\subsection{Hierarchical clustering and Levensthein distance}
We want to show that the Levensthein distance is indeed a \textit{distance}, using the mathematical definition of the term.
\newline
\newline \textbf{Reasoning by the absurd}
\newline If the Levenshtein distance was \textit{\textbf{not}} a general distance, thus according to the mathematical definition:

\[
\begin{cases}
\forall (a,b) \in E^2, d(a,b) \neq d(b,a) \\
\forall (a,b) \in E^2, d(a,b) = 0 \not\equiv  a=b \\
\forall (a,b,c), (d(a,c) > d(a,b) + d(b,c)
\end{cases}
\]
\newline
\textit{Is the Levenshtein distance respecting these criterion, and thus \textit{\textbf{is not}} a general distance ?}
\newline
\begin{itemize}
    \item \textbf{Symmetry}
    \newline Given two strings of characters $a="machine"$ and $b="learning"$ with $(a,b) \in E^2, E={a,b,...,z}$ and $d$ the Levenshtein distance:
    \[
    \begin{cases}
    d(a,b)=d("machine","learning") = 5 \quad\quad ("m","c","h","r","g") \\
    d(b,a)=d("learning","machine") = 5 \quad\quad ("m","c","h","r","g") 
    \end{cases}
    \]
    $$\boxed{d(a,b)=d(b,a)
    }$$
    \item \textbf{Separation}
    Given two strings $a="hello"$ and $b="hello"$, with $(a,b) \in E^2, E={a,b,...,z}$ and $d$ the Levenshtein distance:
    $$\boxed{d(a,b)=d(b,a)=0
    }$$
    \item \textbf{Triangular inequality}
    Given three strings $a="hello"$, $b="hallo"$ and $c="hola$, with $(a,b,c) \in E^3, E={a,b,...,z}$ and $d$ the Levenshtein distance:
    \[
    \begin{cases}
    d(a,c)=d("hello","hola") = 3 \quad\quad ("e","l","a") \\
    d(a,b) + d(b,c) = 1+3=4 \\
    \end{cases}
    \]
    $$\boxed{d(a,c) \ngtr d(a,b)+d(b,c)}$$
\end{itemize}
\textbf{CONCLUSION}: \textbf{Reasoning by the \textit{absurd}}, we clearly observe that all of the three criterias for NOT being a general distance are countered by the Levenshtein distance, hence we can say that the Levenshtein distance is a \textit{\textbf{general distance}}.

\begin{equation} \label{eq4}
\boxed{
Levenshtein \quad distance = 
\begin{cases}
\forall (a,b) \in E^2, d(a,b) \eq d(b,a) \\
\forall (a,b) \in E^2, d(a,b) = 0 \equiv  a=b \\
\forall (a,b,c), (d(a,c) \leq d(a,b) + d(b,c)
\end{cases}
}
\end{equation}
\subsection{Single-linkage criterion}
 Given two clusters $C_1$ and $C_2$, with $C_1 = (x,y), C_2=(x,z) \quad \forall (x,y) \in E^2, E \subset \mathbb{R}$ and $\mathbf{D}(C_1,C_2) = \underset{\mathbf{x}_1 \in C_1,\mathbf{x}_2 \in C_2,}\min d(x_1,x_2)$ being the \textit{\textbf{single-linkage}} criterion
\begin{itemize}
    \item\textit{ Is the Symmetry criterion respected ?}
    \newline $\mathbf{D}(C_1,C_2) = \underset{\mathbf{x}_1 \in C_1,\mathbf{x}_2 \in C_2,}\min d(x_1,x_2) =  0$ but $x_1 \neq x_2$
\end{itemize}
\textbf{CONCLUSION}: Using a counter example, we proved that the single-linkage criterion \textbf{isn't} a \textit{general} distance.
\end{document}